\documentclass[11pt]{article}
\usepackage[utf8]{inputenc}
\usepackage{enumitem}
\usepackage{geometry}
\usepackage{titlesec}
\usepackage{hyperref}
\geometry{margin=1in}
\titleformat{\section}{\normalfont\Large\bfseries}{\thesection}{1em}{}
\titleformat{\subsection}{\normalfont\large\bfseries}{\thesubsection}{1em}{}

\title{CV Building Assistant \\ \large Specification Document}
\author{}
\date{}

\begin{document}

\maketitle

\section{Project Overview}
The CV Building Assistant is an AI-powered conversational tool that enables users to create, refine, and download professional CVs. It interacts with users through a natural language interface to gather relevant information and organize it into a structured CV format, leveraging OpenAI's capabilities.

\section{Core Features}

\begin{itemize}[leftmargin=1.5em]
    \item \textbf{Conversational CV Building} - Progressive input collection across sections such as Personal Info, Education, Experience, etc.
    \item \textbf{CV Template Management} - Multiple exportable and professional layout options.
    \item \textbf{Intelligent Input Handling} - Validation, error feedback, and section skipping.
    \item \textbf{Resume Enrichment} - Smart suggestions using AI for better phrasing and clarity.
\end{itemize}

\section{Agent Responsibilities}

\subsection{cv\_agent.py}
\begin{itemize}
    \item Manages conversation flow and CV section transitions.
    \item Maintains session state.
\end{itemize}

\subsection{conversation\_manager.py}
\begin{itemize}
    \item Handles chat history and memory retrieval.
    \item Supports resuming partially completed sessions.
\end{itemize}

\section{System Modules \& Responsibilities}

\subsection{models/}

\textbf{cv\_data.py}
\begin{itemize}
    \item Represents structured CV content:
    \begin{verbatim}
class CVData:
    personal_info: dict
    summary: str
    education: list
    experience: list
    projects: list
    skills: list
    certifications: list
    references: list
    \end{verbatim}
\end{itemize}

\textbf{conversation.py}
\begin{itemize}
    \item Tracks message metadata (sender, timestamps, content).
\end{itemize}

\subsection{services/}

\textbf{openai\_service.py}
\begin{itemize}
    \item Interface to OpenAI API for response generation and data extraction.
\end{itemize}

\textbf{data\_extraction.py}
\begin{itemize}
    \item Parses structured data from natural language.
\end{itemize}

\subsection{utils/}

\textbf{validators.py}
\begin{itemize}
    \item Ensures input correctness (e.g., phone/email format).
\end{itemize}

\textbf{formatters.py}
\begin{itemize}
    \item Prepares text sections for CV output formatting.
\end{itemize}

\subsection{ui/}

\textbf{streamlit\_app.py}
\begin{itemize}
    \item Initializes Streamlit app and links to backend agents.
\end{itemize}

\textbf{components/}
\begin{itemize}
    \item \textbf{chat\_interface.py}: Displays chat and handles user input.
    \item \textbf{progress\_tracker.py}: Visual indicator of section completion.
\end{itemize}

\section{Development Guidelines}
\begin{itemize}
    \item Use LangChain or AgentExecutor for tool-chaining.
    \item Apply environment management using \texttt{python-dotenv}.
    \item Ensure modularity and testability.
    \item Follow separation of concerns for each module.
\end{itemize}

\section{Testing Scope}

Tests located in the \texttt{tests/} directory include:
\begin{itemize}
    \item \texttt{test\_cv\_agent.py} - CV agent interaction logic.
    \item \texttt{test\_conversation\_manager.py} - Chat memory handling.
    \item \texttt{test\_data\_extraction.py} - Extraction and parsing accuracy.
\end{itemize}

\section{Configuration}

\subsection{settings.py}
\begin{itemize}
    \item Holds application-wide settings (e.g., model name, template path).
\end{itemize}

\subsection{.env.example}
\begin{verbatim}
OPENAI_API_KEY=your-key-here
DEFAULT_MODEL=gpt-4
TEMPLATE_DIR=templates/
DEBUG=true
\end{verbatim}

\section{Future Enhancements}
\begin{itemize}
    \item Multi-language CV generation.
    \item Job-tailored resume creation.
    \item LinkedIn scraping/import.
    \item Export to DOCX or LaTeX.
    \item Cloud-based saving and version history.
\end{itemize}

\end{document}
